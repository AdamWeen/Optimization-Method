\documentclass[UTF8,12pt]{ctexart}
\ctexset{section={name={},number=\chinese{section}}}
\usepackage{fontspec}
\usepackage{xeCJK}
\usepackage{abstract}
\usepackage{geometry}
\usepackage{amsmath}
\numberwithin{equation}{section}%公式按章节编号
\usepackage{graphicx}
\usepackage{subfigure}
\usepackage{indentfirst}
\usepackage{setspace}
\usepackage{newproof}
\usepackage{ntheorem}
\usepackage{bm}%加粗公式的包,在需要加粗的公式前面加\bm
%以下代码以及listings包为添加有中文注释的latex代码的模块
\usepackage{xcolor}
\definecolor{dkgreen}{rgb}{0,0.6,0}
\definecolor{gray}{rgb}{0.5,0.5,0.5}
\definecolor{mauve}{rgb}{0.58,0,0.82}
\usepackage{listings}
\lstset{basicstyle = \small\sffamily,
	frame = single,
	aboveskip=3mm,
	belowskip=3mm,
	showstringspaces=false, 
	columns=flexible,
	framerule=1pt,
	backgroundcolor=\color{gray!5},
	commentstyle=\color{dkgreen},
	keywordstyle=\color{blue},
	commentstyle=\color{dkgreen},
	stringstyle=\color{mauve},
	numbers=left,stepnumber=1,
	numberstyle=\small\sffamily\color{gray},
	breaklines=true,
	breakautoindent=true,
	breakindent=4em, 
	breakatwhitespace=true,
	escapechar = `,
	tabsize=3,
}
\usepackage{algorithm}
\usepackage{algpseudocode}
\usepackage{amsmath}
\usepackage{hyperref}%ref引用,可以定义不同引用的颜色
\hypersetup{colorlinks=true,
            linkcolor=black,
            anchorcolor=blue,
            citecolor=blue}
\usepackage[natbibapa,nodoi]{apacite}

%% 写算法伪代码或者流程的前期准备
\renewcommand{\algorithmicrequire}{\textbf{Input:}}  % Use Input in the format of Algorithm
\renewcommand{\algorithmicensure}{\textbf{Output:}} % Use Output in the format of Algorithm
\renewcommand{\contentsname}{\hspace*{\fill}目\quad 录\hspace*{\fill}}
%跨页伪代码↓
\usepackage{algpseudocode} 
\makeatletter
\newenvironment{breakablealgorithm}
  {% \begin{breakablealgorithm}
   \begin{center}
     \refstepcounter{algorithm}% New algorithm
     \hrule height.8pt depth0pt \kern2pt% \@fs@pre for \@fs@ruled
     \renewcommand{\caption}[2][\relax]{% Make a new \caption
       {\raggedright\textbf{\ALG@name~\thealgorithm} ##2\par}%
       \ifx\relax##1\relax % #1 is \relax
         \addcontentsline{loa}{algorithm}{\protect\numberline{\thealgorithm}##2}%
       \else % #1 is not \relax
         \addcontentsline{loa}{algorithm}{\protect\numberline{\thealgorithm}##1}%
       \fi
       \kern2pt\hrule\kern2pt
     }
  }{% \end{breakablealgorithm}
     \kern2pt\hrule\relax% \@fs@post for \@fs@ruled
   \end{center}
  }
\makeatother

\newtheorem{Definition}{\hspace{2em}定义}
\newtheorem{theorem}{\hspace{2em}定理}
\newtheorem{lemma}{\hspace{2em}引理}
\newtheorem{Proof}{\hspace{2em}证明}%*意为去掉编号
\newtheorem{example}{\hspace{2em}例}
%用于自动替换中文标点为英文符号
\catcode`\。=\active
\newcommand{。}{. }
\catcode`\,=\active
\newcommand{,}{, }

\setlength{\parindent}{2em}%首行缩进
\lstset{extendedchars=false}%解决代码跨页时,章节标题,页眉等汉字不显示的问题

\geometry{left=3.0cm,right=3.0cm,top=2.54cm,bottom=2.54cm}
\title{{\heiti\fontsize{40}{15}\selectfont 最优化方法上机报告}\\[1em]
\fontsize{25}{15}\selectfont \textbf{}\\[7em]}
\author{\fontsize{20}{20}\selectfont \textit{信计}91 \quad \textit{闻逊之}\\[1em]
\fontsize{20}{20}\selectfont \textit{学号}:2193410365\\[2.5em]}
\date{\Large\today\\[1em]\Large {(2021-2022春季学期)}}
\linespread{1.2}
\pagestyle{plain}


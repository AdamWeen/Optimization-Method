\documentclass[UTF8,12pt]{ctexart}
\ctexset{section={name={},number=\chinese{section}}}
\usepackage{fontspec}
\usepackage{xeCJK}
\usepackage{abstract}
\usepackage{geometry}
\usepackage{amsmath}
\numberwithin{equation}{section}%公式按章节编号
\usepackage{graphicx}
\usepackage{subfigure}
\usepackage{indentfirst}
\usepackage{setspace}
\usepackage{newproof}
\usepackage{bm}%加粗公式的包,在需要加粗的公式前面加\bm
%以下代码以及listings包为添加有中文注释的latex代码的模块
\usepackage{listings}
  \usepackage{xcolor}
  \lstset{tabsize=4, %
  frame=shadowbox, %把代码用带有阴影的框圈起来
  commentstyle=\color{red!50!green!50!blue!50},%浅灰色的注释
  rulesepcolor=\color{red!20!green!20!blue!20},%代码块边框为淡青色
  keywordstyle=\color{blue!90}\bfseries, %代码关键字的颜色为蓝色,粗体
  showstringspaces=false,%不显示代码字符串中间的空格标记
  stringstyle=\ttfamily, % 代码字符串的特殊格式
  keepspaces=true, %
  breakindent=22pt, %
  numbers=left,%左侧显示行号
  stepnumber=1,%
  numberstyle=\tiny, %行号字体用小号
  basicstyle=\footnotesize, %
  showspaces=false, %
  flexiblecolumns=true, %
  breaklines=true, %对过长的代码自动换行
  breakautoindent=true,%
  breakindent=4em, %
  escapebegin=\begin{CJK*}{GBK}{hei},escapeend=\end{CJK*},
  aboveskip=1em, %代码块边框
  fontadjust,
  captionpos=t,
  framextopmargin=2pt,framexbottommargin=2pt,abovecaptionskip=-3pt,belowcaptionskip=3pt,
  xleftmargin=4em,xrightmargin=4em, % 设定listing左右的空白
  texcl=true,
  % 设定中文冲突,断行,列模式,数学环境输入,listing数字的样式
  extendedchars=false,columns=flexible,mathescape=true
  % numbersep=-1em
}
\usepackage{algorithm}
\usepackage{algpseudocode}
\usepackage{amsmath}
\usepackage{hyperref}%ref引用,可以定义不同引用的颜色
\hypersetup{colorlinks=true,
            linkcolor=black,
            anchorcolor=blue,
            citecolor=blue}
\usepackage[natbibapa,nodoi]{apacite}

%% 写算法伪代码或者流程的前期准备
\renewcommand{\algorithmicrequire}{\textbf{Input:}}  % Use Input in the format of Algorithm
\renewcommand{\algorithmicensure}{\textbf{Output:}} % Use Output in the format of Algorithm
\renewcommand{\contentsname}{\hspace*{\fill}目\quad 录\hspace*{\fill}}
%跨页伪代码↓
\usepackage{algpseudocode} 
\makeatletter
\newenvironment{breakablealgorithm}
  {% \begin{breakablealgorithm}
   \begin{center}
     \refstepcounter{algorithm}% New algorithm
     \hrule height.8pt depth0pt \kern2pt% \@fs@pre for \@fs@ruled
     \renewcommand{\caption}[2][\relax]{% Make a new \caption
       {\raggedright\textbf{\ALG@name~\thealgorithm} ##2\par}%
       \ifx\relax##1\relax % #1 is \relax
         \addcontentsline{loa}{algorithm}{\protect\numberline{\thealgorithm}##2}%
       \else % #1 is not \relax
         \addcontentsline{loa}{algorithm}{\protect\numberline{\thealgorithm}##1}%
       \fi
       \kern2pt\hrule\kern2pt
     }
  }{% \end{breakablealgorithm}
     \kern2pt\hrule\relax% \@fs@post for \@fs@ruled
   \end{center}
  }
\makeatother

\newtheorem{Definition}{\hspace{2em}定义}
\newtheorem{theorem}{\hspace{2em}定理}
\newtheorem{lemma}{\hspace{2em}引理}
\newtheorem{Proof}{\hspace{2em}证明}%*意为去掉编号
\newtheorem{example}{\hspace{2em}例}
%用于自动替换中文标点为英文符号
\catcode`\。=\active
\newcommand{。}{. }
\catcode`\,=\active
\newcommand{,}{, }

\setlength{\parindent}{2em}%首行缩进
\lstset{extendedchars=false}%解决代码跨页时,章节标题,页眉等汉字不显示的问题

\geometry{left=3.0cm,right=3.0cm,top=2.54cm,bottom=2.54cm}
\title{{\heiti\fontsize{40}{15}\selectfont 最优化方法上级报告}\\[1em]
\fontsize{25}{15}\selectfont \textbf{}\\[7em]}
\author{\fontsize{20}{20}\selectfont \textit{信计}91 \quad \textit{闻逊之}\\[1em]
\fontsize{20}{20}\selectfont \textit{学号}:2193410365\\[2.5em]}
\date{\Large\today\\[1em]\Large {(2021-2022春季学期)}}
\linespread{1.2}
\pagestyle{plain}
